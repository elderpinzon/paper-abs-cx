\section{\label{sec:intro}Introduction\protect}
The scattering of pions off of atomic nuclei has been the subject of extensive study, particularly during the second half of the 20th century, due to its ability to serve as probe of the nuclear structure in the context of effective theories through the understanding of the interactions among mesons and nucleons. The strongest meson-nucleon resonance is the $\Delta(1232)$, and it dominates the sub-GeV energy region. In the case of $\pi^{\pm}-$C interactions this corresponds to $p_{\pi}$ in the 200 to 300 MeV$/c$ range.

The total, elastic and quasi-elastic processes have been measured with $<10\%$ precision by various experiments \cite{Allardyce,Binon,Saunders,Gelderloos,Levenson,Ashery2,Ingram,Jones,Ashery,Bellotti1973,Bellotti1973_2}, however data is scarce for exclusive inelastic processes such as absorption (ABS: $\pi^{\pm}+A\rightarrow (A-N) + N$) and single charge exchange (CX: $\pi^{\pm} + A \rightarrow \pi^{0}+ (A-N) + N$). Moreover, the majority of past experiments measured the combined rate of these two processes, and relied on other experimental results or in theoretical calculations to separate their individual contribution, ignoring possible correlations and systematic uncertainties.

Interest in pion inelastic interactions has increased in recent years due to the use of nuclear targets in GeV-scale neutrino experiments. Neutrinos are primarily detected via charged current quasi-elastic interactions on the target atomic nucleus, generally hydrocarbon or water. The energy of the neutrino, fundamental for the understanding of the oscillation phenomena, is inferred from the measured kinematics of the outgoing lepton. The neutrino-induced single pion production processes also contribute to the cross section in this energy range. If the pion is produced but not detected due to final-state interactions (FSI) within the target nucleus or secondary interactions (SI) elsewhere in the detectors, the inferred neutrino energy will be biased. 

FSI and SI are leading contributors to systematic errors in neutrino oscillation and cross section experiments. Their impact can only be evaluated using predictions based on models implemented in Monte Carlo event generators such as \textsc{Neut}\cite{NEUT}, \textsc{NuWro}\cite{NuWro} and \textsc{Geant4}\cite{bertini}. All these generators use similar implementations of semi-classical cascade models in which the pion is stepped within the nucleus and its fate is calculated following theoretical models where the pion-nucleus interaction is represented by an optical potential. The individual contributions from each channel are calculated from the potential's imaginary parts \cite{Oset,Salcedo}. Precise tuning of the models is achieved through the empirical scaling of the theoretical microscopic interaction rates, relying entirely on the available $\pi$-A scattering data.

Other important scenarios in which $\pi$-A interactions are relevant for neutrino physics are: i) the enhancement of the neutral-current $\pi^{0}$ background in $\nu$-oscillation appearance experiments, and, ii) future pion ring reconstruction capabilities in water Cerenkov detectors via the explicit identification of their hadronic interactions.

An earlier paper from DUET \cite{duet} described our experimental setup and presented a measurement of the combined ABS and CX cross section $\sigma_{\mathrm{ABS}+\mathrm{CX}}$ in the 200 to 300 MeV$/c$ region. In this paper we present separate measurements of $\sigma_{\mathrm{CX}}$ and $\sigma_{\mathrm{ABS}}$ at the same momenta. This was achieved by using a downstream detector to tag forward going photons produced by the decay of a $pi^0$. This measurement will help improve the modeling of FSI and SI in neutrino event generators and to reduce the associated systematic uncertainties on current and future neutrino oscillation and cross section experiments.